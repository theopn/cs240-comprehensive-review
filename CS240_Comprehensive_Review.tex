\documentclass{article}
\usepackage[utf8]{inputenc}

\topmargin=-0.45in
\evensidemargin=0in
\oddsidemargin=0in
\textwidth=6.5in
\textheight=9.0in
\headsep=0.25in

\title{CS240 Comprehensive Review}
\author{Theo Park}
\date{3 May 2022}

\begin{document}

\maketitle

\section{Compiling and Linking}
\subsection{Gcc Flags}
\begin{itemize}
    \item \textit{-c} Compile file into object file
    \item \textit{-g} Debugging symbols
    \item \textbf{\textit{-Wall}} Include ALL Warning
    \item \textbf{\textit{-Werror}} Turn wanings into errors
    \item \textbf{\textit{-O1, -O2, -O3}} Optimize output code
    \item \textbf{\textit{-o filename}} Output to filename
    \item \textit{-ANSI} Adhere to ANSI std
    \item \textit{-std=C99} Adhere to C99 std
\end{itemize}
\subsection{Linking}
Object file contains binary code, symbol tables, and is a compiled form of a C module.
To make it a complete executable, one must link object files, with one of them containing main().

\section{File I/O}
\subsection{Essentials}
\begin{itemize}
    \item FILE *fopen(char *file\_name, char *mode);\\
    Modes are "r", "w", and "a" (append). Returns file ptr on success, NULL on unsuccess, so one must check the return val of fopen().
    \item int fclose(FILE *file\_pointer);\\
    It does not set the file ptr to NULL, so you have to manually set it to NULL. Return val check isn't necessary in this class.
    \item \textbf{int fprinf(FILE *stream, const char *format, \dots);}
    \item \textbf{int fscanf(FILE *stream, const char *format, \dots);}
\end{itemize}
\subsection{Binary File Related}
\begin{itemize}
    \item 
\end{itemize}
\subsection{Key notes with fscanf()}
\begin{itemize}
    \item Utilize \%[] (\%[0-9A-z] \%[\^A-z])
\end{itemize}
\pagebreak


\end{document}